\documentclass{article}
\usepackage{graphicx} % Required for inserting images
\usepackage{enumerate}
\usepackage{enumitem}
\usepackage{amsmath}
\usepackage[a4paper, total={6in, 8in}]{geometry}

\newcommand{\answer}{\textbf{Answer:}$\;$}

\title{Sociology 141 Reading Guide 6}
\author{Alejandro Sanchez Ocegueda}
\date{Februrary 11, 2025}

\begin{document}

\maketitle

\begin{enumerate}[label=\arabic*)]
    \item How does McAdam challenge the view of the role of the \textbf{federal government} in ``traditional liberal accounts'' of the Civil Rights Movement? (pp. 169-170)
    
    \answer 
    The traditional liberal accounts present the federal government as an ally working hard to bring the ideals of the movement to reality.
    McAdam argues that, in fact, that the government tried to remain mostly neutral during this time, as both groups (white supremacists and black people) had significant voting power.
    Instead, the government mostly tried to contain and avoid any excessive disruptions from either side, and was mostly preoccupied with maintaining the public order.
    
    
    \item How does McAdam present the underlying strategy of the Civil Rights Movement? (p. 174)
    
    \answer 
    The strategy employed was, as usual, to disrupt the public order.
    More specifically, however, the strategy that the Civil Rights Movement employed was to deliberately provoke the white supremacists to disrupt the public order \textit{for them}.
    Brilliant stuff!
    
    
    \item According to McAdam, why are \textbf{disruptive tactics} crucial for social movements?
   
    \answer
    
    \item What is \textbf{tactical interaction}?
    
    \answer 
    
    \item What \textbf{3 factors} created a structure of \textbf{political opportunity} for the movement? (p. 737)
    
    \answer
    

    \item What \textbf{indigenous organizations} were crucial for the movement?
    
    \answer
    
    \item According to McAdam's research, what is the relationship between \textbf{tactical innovation} and the \textbf{pace of insurgency}?
    
    \answer

    \item McAdam presents the tactical innovation characterizing the movement between 1955-1970. 
    What are all the \textbf{tactics} he discusses?
    
    \answer

    \item What forms of \textbf{tactical adaptation} develop in response to the \textbf{bus boycotts}?

    \answer


    \item What forms of \textbf{tactical adaptation} develop in repsonse to the \textbf{sit-ins}?

    \answer


    
\end{enumerate}
 




\end{document}
