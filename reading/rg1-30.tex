\documentclass{article}
\usepackage{graphicx} % Required for inserting images
\usepackage{enumerate}
\usepackage{enumitem}
\usepackage[a4paper, total={6in, 8in}]{geometry}

\newcommand{\answer}{\textbf{Answer:}$\;$}

\title{Sociology 141 Reading Guide 3}
\author{Alejandro Sanchez Ocegueda}
\date{January 30, 2025}

\begin{document}

\maketitle

\begin{enumerate}[label=\arabic*)]
    \item What do Piven \& Cloward tell us about the \textbf{relief system} in the U.S. during the Great Depression? (pp. 41-44)

    \answer 
    There are many aspects of the relief system that Piven \& Cloward touch upon.
    Firstly, they say that it was not quite as fleshed out as it is today.
    It also did not cover as much ground as it did today, and places like New York and Philadelphia simply did not have such a thing as aid for the poor (p. 41).
    Furthermore, the aid was given out infrequently, and only to people who for extraordinary circumstances (like disability) could not work, despite the fact that a large part of the population was unemployed.
    
    Another point that they make is that the relief system was deeply associated with shame.
    The people in the care of the relief system were denied the franchise (*don't know what that means*), and they were also looked down upon.
    Part of the job of the relief system was to create a demarcated ``pauper class'' that was so full of shame that poor workers would remain docile and not seek aid, for fear of becoming paupers themselves (p. 42-43).

    These terrible conditions led to several poor and unemployed people mobilizing against the relief system, as described in pages 43 and 44.
    It also sometimes brought about a transformation of consciousness, especially in areas where unemployment was particularly widespread.

    \item How do Piven \& Cloward explain the \textbf{emergence} of the Movement of the Unemployed? (pp. 44-49)

    \answer 

    \item What were the various \textbf{methods} (or tactics) utilized by the Movement of the Unemployed? (pp. 49-60)

    \answer

    \item What is the significance of the \textbf{electoral instability} of 1932 in Piven \& Cloward's account? (pp. 64-68)

    \answer

    \item What \textbf{concessions} were won by the Movement of the Unemployed?

    \answer

    \item In the 2020 interview with Piven, she draws some \textbf{parallels} between our \textbf{contemporary period} and the Movement of the Unemployed in the \textbf{1930s}.
    What do you think are some similarities?

    \answer
    
\end{enumerate}
 




\end{document}
