\documentclass{article}
\usepackage{graphicx} % Required for inserting images
\usepackage{enumerate}
\usepackage{enumitem}
\usepackage[a4paper, total={6in, 8in}]{geometry}

\newcommand{\answer}{\textbf{Answer:}$\;$}

\title{Sociology 141 Reading Guide 3}
\author{Alejandro Sanchez Ocegueda}
\date{January 30, 2025}

\begin{document}

\maketitle

\begin{enumerate}[label=\arabic*)]
    \item What do Piven \& Cloward tell us about the \textbf{relief system} in the U.S. during the Great Depression? (pp. 41-44)

    \answer 
    There are many aspects of the relief system that Piven \& Cloward touch upon.
    Firstly, they say that it was not quite as fleshed out as it is today.
    It also did not cover as much ground as it did today, and places like New York and Philadelphia simply did not have such a thing as aid for the poor (p. 41).
    Furthermore, the aid was given out infrequently, and only to people who for extraordinary circumstances (like disability) could not work, despite the fact that a large part of the population was unemployed.
    
    Another point that they make is that the relief system was deeply associated with shame.
    The people in the care of the relief system were denied the franchise (*don't know what that means*), and they were also looked down upon.
    Part of the job of the relief system was to create a demarcated ``pauper class'' that was so full of shame that poor workers would remain docile and not seek aid, for fear of becoming paupers themselves (p. 42-43).

    These terrible conditions led to several poor and unemployed people mobilizing against the relief system, as described in pages 43 and 44.
    It also sometimes brought about a transformation of consciousness, especially in areas where unemployment was particularly widespread.

    \item How do Piven \& Cloward explain the \textbf{emergence} of the Movement of the Unemployed? (pp. 44-49)

    \answer
    Put simply, the Movement of the Unemployed came about as a consequence of the rapid deterioration in the quality of life of the millions of people who were suddenly left without a job during the Great Depression (pp. 44-48), along with a new collective consciousness that blamed the system rather than the individuals affected (p. 49).

    In a period of only two years, millions of Americans suddenly found themselves out of a job and thrust into poverty.
    These newly-unemployed people began to experience a sudden increase in hardships like hunger, the shame of unemployment, widespread divorces, desertions, poverty, and drunkenness.
    
    Furthermore, the narrative being put forth by the government and businesses was one of minimizing the problem and of blaming the citizens for their own misfortunes (p. 47, 49).
    Due to the individualistic ideologies that are prevalent in American society, this narrative was believed for a long time.
    As the depression worsened and more and more people became unemployed, however, the attitude of the unemployed also began to change.
    In Piven \& Cloward's words:
    \begin{center}
        \textit{[The unemployed] began to define their personal hardship not just as their own individual misfortune, but as misfortune they shared with many of their own kind.
        And if so many people were in the same trouble, then maybe it wasn't they who were to blame, but ``the system'' (p. 49)}
    \end{center}

    \item What were the various \textbf{methods} (or tactics) utilized by the Movement of the Unemployed? (pp. 49-60)

    \answer
    There were several methods that the Movement of the Unemployed used:
    \begin{enumerate}
        \item 
        \textbf{Mob lootings:}
        Lootings consisted of raiding markets, delivery trucks, and other establishments, mostly to obtain food supplies.
        What makes these interesting is that the organized looting of food became a nation-wide phenomenon (p. 49).
        \item
        \textbf{Marches and political demonstrations:}
        These mostly consisted of large numbers of unemployed people (in the thousands) who would walk out into the street and protest the unfairness of their living conditions.
        A large number of demonstrations happened on March 6, 1930, which was declared 
        Some of these marches were peaceful (e.g. the one in San Francisco), but others often broke out into violent conflict between protesters and the police.
        The bloodiest encounter was in New York, where the police brutally attacked those who protested.
        \item
        \textbf{Rent riots:}
        These consisted of impromptu gatherings of people to prevent families who had been evicted from being physically removed from their homes.
        They were prevalent in Chicago.
        They were often met with arrests, beatings, and even killings (p. 55), but they were also effective in letting tens of thousands of people keep their homes (p. 54), and they even forced relief officials to give out money for rent payments (p. 55).
        \item \textbf{Relief insurgency:}
        These consisted of masses of people asking for relief, and in collectively protesting and applying mass pressure to the relief offices whenever someone was denied their relief (pp. 56-57).
        These were often organized and support by the Unemployed Councils of the city.
        
    \end{enumerate}
    
    \item What is the significance of the \textbf{electoral instability} of 1932 in Piven \& Cloward's account? (pp. 64-68)

    \answer
    The electoral instability resulted in a massive shift in power, from the Republican party to the Democratic party.
    The Republicans had been in power since 1920, but the dire circumstances led to the election of Franklin Delano Roosevelt in 1932.
    This was significant because FDR campaigned by promising to put people at the bottom of the economic pyramid first.
    Not only that, he was the one to finally give concessions to the unemployed by providing relief and employment.

    \item What \textbf{concessions} were won by the Movement of the Unemployed?

    \answer
    The concessions that were won by the Movement of the Unemployed were the Federal Emergency Relief act, the Civilian Conservation Corps, and the Public Works Administration.
    Concretely, these initiatives provided the unemployed with relief and jobs, which helped them improve their living situations.    

    \item In the 2020 interview with Piven, she draws some \textbf{parallels} between our \textbf{contemporary period} and the Movement of the Unemployed in the \textbf{1930s}.
    What do you think are some similarities?

    \answer
    There are a couple of similarities I notice.
    For one, there is a rising rate of unemployment, and a worsening of living conditions.
    But more importantly, people are \textit{taking notice};
    as Piven points out, ever since the Occupy movement, people have been quite aware and dissatisfied with the extreme levels of wealth inequality.
    
    Moreover, I have personally noticed that there is a transformation of consciousness going around, particularly online.
    This may just be due to the kinds of websites I visit, but I do see a lot more people who begin to blame capitalism or other structural factors, for the deteriorating quality of living conditions, as opposed to the hyper-individualistic mindset that I saw a couple of years ago.
    Also, the assassination of Brian Thompson showed very clearly that people are sick and tired of the predatory and abusive practices of businesses towards the working class.
    
\end{enumerate}
 




\end{document}
