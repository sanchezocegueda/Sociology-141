\documentclass{article}
\usepackage{graphicx} % Required for inserting images
\usepackage{enumerate}
\usepackage{enumitem}
\usepackage[a4paper, total={6in, 8in}]{geometry}

\newcommand{\answer}{\textbf{Answer:}$\;$}

\title{Sociology 141 Reading Guide 1}
\author{Alejandro Sanchez Ocegueda}
\date{January 23, 2025}

\begin{document}

\maketitle

\begin{enumerate}[label=\arabic*)]
    \item What Is Piven \& Cloward's \textbf{main argument} in this chapter? (p. 3)

    \answer The main argument that Piven \& Cloward make in this chapter is that while it is widely believed that protest is the only choice that the poor have to gain some power in a system that discriminates and marginalizes them, that is not really the case.
    In fact, Piven \& Cloward argue that protests are not a matter of free choice, and that sometimes protest is not even an accessible to lower-class groups.
    Furthermore, the conditions under which the lower class can protest, efficacy of these protests, and the forms that it must take are all constrained by the underlying social structure in which they happen.
    Additionally, this social structure often makes these protests less widespread and less effective.

    \item How do they characterize the \textbf{transformation of consciousness} that occurs with the emergence of a protest movement? (p. 3-5)

    \answer They give three criteria that must be met for a transformation of consciousness:
    \begin{enumerate}
        \item The system must lose some legitimacy;
        that is, a large number of people come to believe that the authority of the rulers and the legitimacy of the arrangements is unjust and wrong.
        \item The people who are usually fatalistic (meaning that they believe that there is no way that the system might change) begin demanding ``rights'' that entail a demand for change.
        \item People who believe that they are helpless recognize that they have some power to change their situation and, by extension, the system.
    \end{enumerate}

    \item How do they characterize the \textbf{transformation of behavior} that occurs with the emergence of a protest movement?    

    \answer
    The authors give two criteria for the transformation in behavior:

    \begin{enumerate}
        \item Masses of people must become defiant. 
        In other words, a large number of people must break and challenge the traditions and laws that they generally follow.
        \item The defiance of people must be acted out as a collective.
        These can come in the form of riots and protests.
        However, they can also be individual acts, so long as the actors perceive themselves to be part of a larger group with which they share a common set of protest beliefs.
    \end{enumerate}

    \item Piven \& Cloward argue that protest movements are usually ``structurally precluded'' but that they do emerge under exceptional conditions.
    What are the \textbf{exceptional conditions} that allow for the \textbf{emergence} of movements?
    \textit{(Note that they forward their argument by presenting and critiquing other prevailing theories.)} (p. 6-14) 

    \answer The exceptional conditions are rapid and extraordinary disturbances in the larger society.
    They can be either economic---both improvement and deterioration of people's economic status---or they can be a shift in perspective of the social norms around them---the rules go from being just and immutable to suddenly becoming unjust and mutable.
    In either case, Piven \& Cloward's main point is that for the poor to uprise, the change must be \textit{sudden}.
    It is not sufficient for society to be unjust or for people's economic situations to change---the change must be noticeable.
    
\end{enumerate}
 




\end{document}
