\noindent \textbf{Lecture 4 --- January 30\th}

\subsubsection{On Striking}
A strike is a tactic where people withhold labor.
Typically, strikes happen at the level of one company.
There are also industry-wide strike, in which workers of a given industry withhold labor.
A general strike is an extreme case of this.
In a general strike, \textit{all} workers within a general area go on strike.
We

\section{The Movement of the Unemployed}
We now want to pay attention to how the Movement of the Unemployed illustrates Piven \& Cloward's theory of social movements.

\subsection{Emergence}
There were several notable components of P\&C's theory that are visible in the Movement of the Unemployed.
We describe them below.

\subsubsection{Massive social and economic changes}
This one is quite obvious.
The Great Depression came with drastic changes to the living conditions of everyone.
Many people became poor.


\subsubsection{Institutional Breakdown}
Regarding the institution of the family, the number of the divorces skyrocketed.
There was a massive increase in homelessness, indicating the breakdown of the institution of the home.
Malnutrition and disease rates.
Suicide rates also went up.

\subsubsection{Transformation of Consciousness}
People began to recognize the failures came at a systemic level.

Part of the reason why this was so crucial in the United States was because of the stigma and shame that poverty entailed.
Economic failure was deemed as a matter of personal failure.
However, the radical shift was from individual shame to collective indignation.

\subsubsection{Division \& competition among elites}
The very first division starts to happen between political officials at the local and federal level.

\subsubsection{Parallels to the Pandemic}
There are many parallels to the Great Depression and contemporary times.
These were unprecedented times.
Several industries shut down, people were being evicted, children could not go to school, etc.

\subsection{Objectives}
The main objective of the movement was to elicit a reaction from the government.
Specifically, they wanted to force concessions from the government, mainly relief.

\subsection{Strategies and tactics}
The main \textbf{strategy} of the Movement of the Unemployed was that of disrupting social order.
To incite disruption, they had several tactics.
These included but were not limited to: looting, rent riots (eviction defenses), and mass protests.
Another cool tactic was to reinstall water, gas, and electricity services after it was cut off by companies like PG\&E.
Relief insurgency (occupation) was done to try to force the hand of the people in power.

\noindent \textbf{Strategy:} Plans for achieving some objective of a social movement. 

\noindent \textbf{Tactic:} Methods (forms of action) through which the strategies are enacted.


