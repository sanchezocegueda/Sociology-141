
\vspace{3mm}
\noindent Lecture 17 --- April 1\st

\section{Emergence}
How can we explain the emergence of the movement?
One way we can understand the emergence of the movement is to understand the economic context in which it emerged.
There were several economic factors that set off the Occupy movement:
\begin{enumerate}
    \item The Great Recession: a lot of deregulation happened that led to an economic downturn.
    \item \textbf{Obama Disillusionment:} There was a growing sense of disillusionment with electoral politics after the Obama presidency.
\end{enumerate}


\subsection{Obama Disillusionment}
People were disappointed with the lack of change following Obama's election.

Another big 

\subsection{Arab Spring}
In the Spring of 2011, there was a series of uprisings in Arab countries.
They started in Tunisia and spread to Egypt and Lybia, etc.
In some of these cases, the people actually succeeded in overthrowing these regimes.
This served as an enormous inspiration globally, particularly because these were some of the most repressive regimes in the world.
This was achieved by massive amounts of people occupying public squares.

More broadly, the tactic of occupying spaces was spreading globally.
These movements were illustrating that when people collectively asserted their presence in public spaces, they couldn't easily be ignored. 

\section{Anarchism}
Anarchism can be thought of as a rejection of all forms of domination or coercion.

This ideology has an emphasis on practice.
Means should reflect or embody the ends.
This can be thought of as ``building a new society in the shell of the old.''
In the context of the Occupy movement, this was enacted by adopting horizontalism; that is, having no hierarchy, leaders, or representatives.

The rejection of coercion was embodied by the use of consensus-based democracy.
This is in opposition to the majority-rule democracy that we have in the US today.
 
Finally, the rejection of the state was present as the use of direct action.
In other words, people would do things instead of demanding that the government do something.
This is why Occupy Oakland did not see themselves as a protest; a protest would be asking the government to do something.
They also used mutual aid, which is the strategy of sharing resources from within the community to meet the community's needs.