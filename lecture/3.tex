\newpage
\noindent \textbf{Lecture 3 --- January 28\th}

% The main argument of Piven \& Cloward's work is that possibilities for protest movements are \textbf{structurally shaped and constrained}.
% The defining feature of a movement is \textbf{collective defiance}.

% Institutions of society begin to strain and break down.

\section{Methods of Protest Movements}
How can we understand the different methods and tactics employed by different movements?
Why do some people block freeways, while others burn down police stations, and others peacefully take to the streets to march?
Moreover, why don't people try to work \textit{with} the system instead of \textit{against} it?

\subsection{Strategy: Electoral Politics}
It turns out that the very first step to a protest or social movements is actually a change in electoral politics.
In other words, every social movement\footnote{At the very least, every social movement that we cover in Sociology 141.} has been preceded by a notable change in voting patterns.
This indicates that people \textit{do} indeed turn to the electoral system before breaking out of it.
However, Piven \& Cloward also observe that, after enough empty promises and unmet needs, people can (and will) pursue their goals through non-electoral means.

\subsection{Factors}
What determines the different methods that people use?

\begin{enumerate}
    \item 
    \textit{Concrete experiences of daily life:}
    These shape grievances and determine the targets of the protest.
    For instance, people will not rebel against capitalism, they will rebel against their employer who treats them unfairly.
    Similarly, the unemployed will not protest against social policy, but rather against the relief bureaucracy that directly affects their daily lives.
    \item 
    \textit{Structuring of collectivities:} 
    This refers to how people are aggregated.
    For instance, factory workers are aggregated in a factory, where they can share their experiences and collude.
    On the other hand, 
    \item 
    \textit{Disruptive impact:}
    This is determined by structural positioning.
    % TODO: rewrite this bc it's wordy and hard to read
    Defiance is only effective insofar as it disrupts the institutions against which they are.
    Stated differently, there must be some threat of interrupting the status quo on part of the movement.
    Disruptive impact can come about through withholding labor, consumption, by rioting, and many, many others.
\end{enumerate}
In the sections that follow, we will spend some time working on specific strategies .

\subsection{Strategy: Disruption}
We define \textbf{disruption} as collective defiance (or withholding of cooperation) to cause an interruption to the regular flow of institutional operations.
As long as the institution does not provide change, social movements will resort to disrupt the system until some change comes about.
At its core, every social movement has some aspect of disruption.
that has brought about change has done so through \textit{some} form of disruption.
Disruption forces, at the very least, that the system pays attention, and it often also forces the system to change.

One form of disruption is the \textbf{riot}.
Riots can be conceived of as a \textit{disruption of public order}.
They are often shamed by the public view and dismissed as   ``irrational.''
But, as it turns out, riots are in fact highly rational actions taken by people who are so oppressed by the system that they cause disruption in the one place that they have power:
maintaining the public order.

A common criticism about movements is that they are ``too disruptive.''
This point of view misses the entire \textit{point} of social movements.
To put it differently, the people complaining about the Palestine protests at graduation have a serious fundamental misunderstanding of what a protest is even supposed to be.
% Riots disrupt public order.
% A rebellion is a more general term that describes the movement itself.


\begin{center}
    Collective defiance $\Rightarrow$ Institutional Disruption $\Rightarrow$ Political Reverberations (concessions).
\end{center}