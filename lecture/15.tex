\vspace{3mm}
\noindent Lecture 15 --- March 18\th

\section{Resistance and Repression}
What is the relationship between repression and resistance?

\subsection{Repression}
There were many tactics of repression on part of COINTELPRO, the FBI's program to.
\begin{itemize}
    \item Infiltration
    \item Arrests
    \item Assassinations
\end{itemize}
This repression intensified dramatically after the election of Richard Nixon.

\subsection{Resistance}
Repression breeds resistance.

\section{The Chicano Movement}

\subsection{The 1940s and 1950s}
These `movements' were mostly separate organizations.


Historically, in the US, Mexican-Americans were legally classified as ``white.''
This was a consequence of US military invasion and conquest, in the Mexican-American War (1846-1848).
The war concluded with the treaty of Guadalupe Hidalgo, which ends with the Mexican Cession, as it was forced to cede almost half of its territory to the United States.

A lot of (former) Mexicans were ascribed US citizenship.
This gave them the legal classification of ``white,'' even though they were heavily discriminated against.
In many ways, they faced the same obstacles as Black people during Jim Crow, but the legal cases against this were always rejected due to the fact that they were technically white the whole time.

\subsection{Repression}
There were several repression.

Repression was a big part of the Chicano Movement.
Highly visible forms of repression and police violence fueled the resistance.
It increased support fo the movement, radicalized and politicized people.
Other forms of repression, such as infiltration and propaganda campaigns, were more effective at undermining the movement.

McAdam --- the structure of political opportunity changed from 1968-1969 to 1970-1971.