
\vspace{3mm}
\noindent \textbf{Lecture 11 --- February 25\th}

Students became politically organized.
They took on the objective of civil rights.
In the bay area, this was specifically targeted towards discriminatory hiring practices.
All the organizing is happening 1963-1964.

The FSM in its entirety was in the fall semester of 1964.
The Bay area civil rights movement happened the summer before that.

A range of students come together under the name of the United Front.
They firest set up a few tables in front of Sather gate as their fierst act of vivil disobedience.
This provoked some response by the university adminisration --- they are issued citations.
But they were alsmot immediately replaced by another group of students.

At the end of the day, what happens is that the university suspends some students to make an example out of them.
This backfires --- rather than discouraging them, it further incites them to protest.
They set up around 15 tables in the middle of Sproul plaza.
When the university tries to give them citations, they refuse to identify themselves.


\section{University's Response}
One police car tries to arrest Jack Weinberg.
Students respond by creating a blockade and stopping the police from arresting Jack Weinberg.
The blockade lasted 32 hours.

The university attempted to negotiate with them.
The students would leave peacefully, and let the police car go.

The university also agrees on continuing to negotiate with students on free speech.

Students were free to speak about any political issues, but they were not allowed to advocate for any sort of political action.

The students were employing the tactic of going limp to make it as hard as possible to be dragged out of the building.
The idea was that if it took enough time to kick them out of the building, the students in the morning could see the violence on part of the police, which would cause outrage.
