% Date: January 21st

\counterwithout{section}{chapter}
\noindent \textbf{Lecture 1 --- January 21\st}

\section{Approaches to Social Movements}
In this first section, we first give a brief overview of the different approaches to studying social movements and political action.


\subsection{The Pluralist Approach}
% TODO
The argument of the pluralist was that the US was good because the system included a plurality of interests (hence the name).
The idea was that even if there were imbalances and inequalities, the political power was widely distributed between competing groups;
that is, no one group completely controls the system.


\subsection{The Classical Approach}

The classical approach refers to the way scholars and sociologists studied social movements before the 1970s.

Fun fact: the course is titled \textit{Social Movements \& Political Action}.
In the classical approach.

Aside: Political Action.
Before, the study of social movements was relegated to social psychology, mostly because it was viewed as an irrational action.
It was categorized as a form of ``deviance.''
They were viewed as ``pathological'' or ``irrational''.

% \section{Political Sociology}

Political Science and Political Sociology studied Institutional Politics, whereas Social Psychology studied Insurgent Politics

\textbf{Institutional Politics:} Politicis within formal institutional channels.
For instance, electoral processes.
During the classical times, only institutional politics were viewed as ``real'' politics. 
Other forms of politics (often known as insurgent politics) were viewed as ``not real'' politics.

\noindent \textbf{Insurgent Politics:} Collective behavior (?)

Pluralist approach $\Rightarrow$ Classical approach

The classical approach is based on the pluralist approach to politics.


The underlying assumption is that the competition between the different political groups would balance out the disparity among them.
This, in theory, ensured that the political system was
\begin{enumerate}
    \item \textbf{Open:} no group monopolizes power to block other groups.
    \item \textbf{Responsive:} the need for coalitions necessitates responding to the demands of other groups.
\end{enumerate}

This was the dominant theory during the 1960s.

C Wright Mills refers to this as ``Elite Theory.''

\subsection{The Elite Theory Approach}
This is a new approach first proposed by C. Wright Mills, and it centers around the concept of the ``elite.''

\noindent \textbf{Elite:} those who hold power in a society.
The idea transitioned from the Elite Theory to the Resource Mobilization approach.

The postulates of Elite Theory are as follows:
\begin{enumerate}
    \item The system is controlled by the elite
    \item Most groups are politically excluded
\end{enumerate}

\subsection{The Resource Mobilization Approach}
This new approach is based on the Elite Theory approach to politics. 

\subsection{Why care about all these approaches?}
Today, all of this falls under the ``Political Action'' umbrella term.
Furthermore, the classical approach has been completely rejected.
But why is it worth our time?
In part, it is important to know about these things because understanding these old approaches gives you a sense of what newer scholars are arguing against.
Most importantly, it is important to understand how social movements were often cast to the wayside and treated as irrational and deviant.
These ideas and assumptions, antiquated and refuted as they are, continue to shape the way people think and approach social movements to this very day.


\section{Social Movements}
There are several questions we will address throughout this course

\begin{enumerate}
    \item Emergence: under what conditions do mass social movements to emerge?
    How can we learn to recognize those conditions so that we can fully exploit them when they show up?
    \item Objectives:
    What have social movements sought to achieve?
    \item Strategies/Tactics: What different kinds of tactics have been used to attain those objectives? 
    Which strategies and tactics have been effective?
    What factors influence the efficacy?
    \item Organization:
    How have different movements organized themselves?
    What different forms of organization have they adopted?
    \item Challenges:
    What kinds of challenges have different movements faced?
    What different state oppression have movements been challenged with?
    How have different movements responded or adapted to those challenges?
    \item Impact:
    How can we fully assess the impacts of movements?
    What have been their intended impacts?
    What have been the unintended impacts or reverberations of social movements?
\end{enumerate}

