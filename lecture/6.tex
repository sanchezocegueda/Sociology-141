\vspace{3mm}
% \noindent \textbf{Lecture 6 --- February 6\th}
\setcounter{section}{0}




\section{Emergence of Movement (Insurgency)}
This is a classical political theory book by Doug McAdam.
In this text, McAdam argues that there are some conditions that must be met before the rise of a movement:
\begin{enumerate}
    \item Structure of Political Opportunity
    \item Indigenous Organizational Strength
    \item Cognitive Liberation
\end{enumerate}

\subsection{Structure of Political Opportunity}
The alignment of groups within the larger political environment.
Another way to put it is as  ``a shift in the balance of power relations'' at any given moment.
There are moments when the balance of power relations is shifted, creating an open for insurgency.
Insurgents have leverage or power that they didn't have before.
Alternatively, the ability of the elites to repress the movement freely, the ``cost of repression,'' if you will, is diminished.

\subsection{Indigenous Organizational Strength}
The concept of \textit{Indigenous Organizational Strength} can be conceived of as the organizational readiness to exploit political opportunity.


\subsection{Cognitive Liberation}
The collective liberation of the movement...
This is almost the same concept as Piven and Cloward's ``transformation of consciousness.''

\section{Development/Decline of Movement}

\subsection{Structure of Political Opportunity}
The balance of power relations is dynamic.
\begin{enumerate}
    \item Structure of Political Opportunity
    \item Organizational Strength
    \item Collective Attribution
    \item Social Control Responses
\end{enumerate}

\subsection{Organizational Strength}
In the long run, indigenous organizations are not enough to sustain a movement.
In fact, he argues (in opposition to P\&C) that formal organizations are necessary for centralized coordination.
However, these centralized formal institutions come with dangers:
\begin{itemize}
    \item Oligarchization
    \item Co-optation
    \item Disconnection from indigenous base
\end{itemize}


Civil rights organizations
\begin{itemize}
    \item NAACP - National Association for the Advancement of Colored People
    \item CORE - Congress of Racial Equality
    \item SCLC - Southern Christian leadership Conference
    \item SNCC - Student Non-Violent Coordinating Committee
\end{itemize}
