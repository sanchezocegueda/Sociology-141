\vspace{3mm}
\noindent \textbf{Lecture 8 --- February 13\th}


\section{Civil Rights Movement}

\begin{itemize}
    \item[[ 1954]] Brown vs Board of Education, Muder of Emmett Till
    
    \item[[ 1955]] Montgomery Bus Boycott

   \item[[ 1960]] Grensboro Sit-ins
   \item [[ 1961]] Freedom Rides
\end{itemize}


\subsection{Emmett Till (1954)}
A 14-year-old black boy who was falsely accused of flirting with a 21-year-old white woman.
He was beaten to death and then his body was thrown in to a river.

\subsection{Boycott}
A boycott is a tactic in which individuals abstain from consumption.
It is similar to a strike, in which workers withhold labor.
Here, consumers withhold consumption.
The idea is to cause economic disruption.

\subsection{Sit-in}
A sit-in is the organized occupation of space and refusal to leave until demands are met.

Form of Occupation -- physically taking over space in a manner that lays claim to and prefigures new uses for it.

\subsection{Freedom Rides}
Sending biracial groups (freedom riders) to ride buses throughout the south.
They were going around challenging segregation in those facilities.
The purpose was to place pressure on the government to force them to act on the ruling.
Their aim was economic disruption through the segregationist violence.



\subsubsection{Adaptations}
There were several adaptations that resulted.
On the legal end, the governments would give citations/arrests to boycotters.
On the extralegal end, there was a lot of violence (beatings, bombings, and more).


In 1960, JFK defeats Richard Nixon by less than $\frac{1}{3}$ of the popular vote:
$49.7\%$ vs $49.5\%$.

\section{Strategy of Direct Action}
Instead of acting through elected officials, acting outside of formal institutional politics in order to achieve the group's objectives.


\section{Strategy of Civil Disobedience}
A strategy of intentionally breaking the law.
