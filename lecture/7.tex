\vspace{3mm}
\noindent \textbf{Lecture 7 --- February 11\th}

\subsection{Collective Attribution}
This is basically the same as Piven \& Cloward's concept of ``cognitive liberation.''

\subsection{Social Control Responses}
We are taught that in the traditional liberal view, the government is an ally working in favor of the goals of the movement.
McAdam challenges this view, and instead says that if anything, the government is attempting to maintain a neutral position and avoid antagonizing either side too much.

% \section{Nonviolence}
% This is a concept that is perhaps antithetical to the point of social movements.

\section{Tactical Innovation}
\subsubsection{Black Insurgency (1955-1970)}

\begin{enumerate}
    \item \textbf{Structure of Political Opportunity:}
    One big evolution along this axis is that Black insurgents found themselves in a position where they had more leverage, while the cost of repression also became higher.
    \begin{enumerate}
        \item $\uparrow$ Black electoral power.
        \item Collapse of southern cotton economy $\rightarrow$ urban migration.
        \item $\uparrow$ Significance of Third World\footnote{In Cold War lingo, Third World countries were countries that were not aligned with NATO or the Warsaw Pact.} to US foreign policy (Cold War)
        
    \end{enumerate}
    \item \textbf{Indigenous Organizational Strength:}
    \begin{enumerate}
        \item Black Churches
        \item Black Colleges
        \item Southern Wing of the NAACP
    \end{enumerate}
    \item \textbf{Cognitive Liberation:}
    This is not explicitly mentioned by McAdam in the reading.
    However, McAdam does mention that the insurgents must feel that ``the time is right.''
    \item \textbf{Social Control Responses:}
    These shape the movement's ability to either develop or decline.
\end{enumerate}

\section{Significance of Disruptive Tactics}

Challengers lack institutional power $\Rightarrow$ Must resort to non-institutional tactics that force a response.

\begin{enumerate}
    \item Bus boycotts
    \item Sit-ins
    \item Freedom rides
    \item Community campaigns
    \item Riots
\end{enumerate}

\section{Tactical Interaction}

Tactical Innovation: the creativity of insurgents in devising new tactical forms.

Tactical adaptation: the ability of opponents to neutralize these moves on 

% TODO: draw circle of tactical innovation and tactical adaptation

